%jsarticleから移行, 2024/3/3
%\documentclass{jlreq}
%\usepackage{luatexja}
%-----グラフィック------------------------------------------------------

%画像(xcolorつけたらjpgも表示された)なお画像ファイル名は英字にしないとエラーを吐くことに注意
%\usepackage{graphicx, xcolor}
%tcolorboxのおかげでgraphicx, xcolorが読み込まれる
\usepackage{tcolorbox}
%tikzライブラリ
\usepackage{tikz}
\usepackage{circuitikz}
\usetikzlibrary{intersections,calc,arrows.meta,patterns}
\usetikzlibrary{decorations.markings,decorations.pathmorphing,shapes.geometric,positioning}
%ページまたぎを実現/拡張
\tcbuselibrary{breakable,skins,theorems}
%自分の, グレーの枠
\newtcolorbox{mybox}{colframe=gray!25, colback=gray!25, sharp corners}
\newtcolorbox{mybox2}{colframe=gray!10, colback=gray!10, sharp corners}
%サブフォルダ内の図へのPATHを通す, {{}}内にフォルダ名を記入
%\graphicspath{{}}
%svgファイル



%-----以下数式関係----------------------------------------------------

% 数式, 米国数学会が開発したのがamsmath, フォントを使うためのパッケージがamssymbs, 何故かamsfontsまで履いていたので削除(2024/3/3追記)
\usepackage{amsmath,amssymb}
%定理環境、コピペ(2024/3/3)
\usepackage{amsthm}
\begin{comment}
\theoremstyle{definition}
\newtheorem{dfn}{定義}
\newtheorem{prop}{命題}
\newtheorem{lem}{補題}
\newtheorem{thm}{定理}
\newtheorem{cor}{系}
\newtheorem{rem}{注意}
\newtheorem*{rem*}{注意}
\newtheorem{fact}{事実}
\newtheorem{e.g.}{例}
\newtheorem{prob}{演習問題}[section]
\end{comment}
\usepackage{thmtools}
\declaretheorem[name=問題, numberwithin=section]{prob}
%\renewcommand{\qedsymbol}{$\square$}
% 花文字?(2023/6/10追加), RSFS, Ralph Smith's Formal Script(2024/3/3追記)
\usepackage{mathrsfs}
% 数式太文字
\usepackage{bm}
%proofカスタム
\renewcommand{\proofname}{\textbf{証明}}
%参照した数式にのみ数式番号を振ろう(2023/10/6)
\usepackage{mathtools}
\mathtoolsset{showonlyrefs=true}
% 消えた項を明記(2023/7/26追加)
\usepackage[thicklines]{cancel}



%-----以下自然科学関係------------------------------------------------------

%便利なので
\usepackage{physics}
%単位
\usepackage{siunitx}
%元素とか, 化学反応式とか
\usepackage[version=4]{mhchem}


%-----以下図表の配置関係------------------------------------------------------

% 表の強制固定
\usepackage{float}
%captionに*つけて「図」を消す(2023/10/7)
\usepackage{caption}


%-----以下その他------------------------------------------------------

% comment
\usepackage{comment}
%新出用語のためのマクロ, 2024/3/3, 美文書p.76-77
\newcommand{\term}[1]{{\sffamily\bfseries #1}}



\usepackage{txfonts} %設定部分
\usepackage{listings}

%\renewcommand{\lstlistingname}{リスト} % 「ソースコード」を変更する
\lstset{language=C,% ソースの種類の指定
        basicstyle=\footnotesize,% リスト全体の設定
        commentstyle=\color{magenta}\textit,% コメント部分の設定
        keywordstyle=\textbf,%  C言語の予約語(if,for,while等)の設定
        %keywordstyle=\color{red}\bfseries,%
        classoffset=1,%
        breakindent=20pt,%    改行時インデント量。デフォルト:20pt。
        breaklines=true,%   行が長くなってしまった場合の改行。
        frame=tb,framesep=7pt,% frame は top,left,right,bottom の1文字で指定、大文字は二重線
        showstringspaces=false,% string 中のスペースを記号表示するか
        numbers=left,% 行番号を付ける位置
        stepnumber=2,%  何行ごとに行番号を表示するか デフォルトは 1
        numberstyle=\scriptsize\color{black}% 行番号の表示スタイル
        }%

%-----以下 hyperref 関係------------------------------------------------------
\usepackage[
setpagesize=false,%
bookmarks=true,%
bookmarksdepth=tocdepth,%
bookmarksnumbered=true,%
colorlinks=false,%
allcolors=blue%
pdftitle={spectrum of atom and molecule},%
pdfsubject={report for the class of student experiment II},%
pdfauthor={Yuichi Mori},%
pdfkeywords={}%
]{hyperref}
\hypersetup{
colorlinks=true,
citecolor=magenta,
linkcolor=blue,
urlcolor=magenta,
}